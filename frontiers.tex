%%%%%%%%%%%%%%%%%%%%%%%%%%%%%%%%%%%%%%%%%%%%%%%%%%%%%%%%%%%%%%%%%%%%%%%%%%%%%%%%%%%%%%%%%%%%%%%%%%%%%%%%%%%%%%%%%%%%%%%%%%%%%%%%%%%%%%%%%%%%%%%%%%%%%%%%%%%
% This is just an example/guide for you to refer to when submitting manuscripts to Frontiers, it is not mandatory to use Frontiers .cls files nor frontiers.tex  %
% This will only generate the Manuscript, the final article will be typeset by Frontiers after acceptance.   
%                                              %
%                                                                                                                                                         %
% When submitting your files, remember to upload this *tex file, the pdf generated with it, the *bib file (if bibliography is not within the *tex) and all the figures.
%%%%%%%%%%%%%%%%%%%%%%%%%%%%%%%%%%%%%%%%%%%%%%%%%%%%%%%%%%%%%%%%%%%%%%%%%%%%%%%%%%%%%%%%%%%%%%%%%%%%%%%%%%%%%%%%%%%%%%%%%%%%%%%%%%%%%%%%%%%%%%%%%%%%%%%%%%%

%%% Version 3.4 Generated 2022/06/14 %%%
%%% You will need to have the following packages installed: datetime, fmtcount, etoolbox, fcprefix, which are normally inlcuded in WinEdt. %%%
%%% In http://www.ctan.org/ you can find the packages and how to install them, if necessary. %%%
%%%  NB logo1.jpg is required in the path in order to correctly compile front page header %%%

\documentclass[utf8]{FrontiersinHarvard} % for articles in journals using the Harvard Referencing Style (Author-Date), for Frontiers Reference Styles by Journal: https://zendesk.frontiersin.org/hc/en-us/articles/360017860337-Frontiers-Reference-Styles-by-Journal
%\documentclass[utf8]{FrontiersinVancouver} % for articles in journals using the Vancouver Reference Style (Numbered), for Frontiers Reference Styles by Journal: https://zendesk.frontiersin.org/hc/en-us/articles/360017860337-Frontiers-Reference-Styles-by-Journal
%\documentclass[utf8]{frontiersinFPHY_FAMS} % Vancouver Reference Style (Numbered) for articles in the journals "Frontiers in Physics" and "Frontiers in Applied Mathematics and Statistics" 

%\setcitestyle{square} % for articles in the journals "Frontiers in Physics" and "Frontiers in Applied Mathematics and Statistics" 
\usepackage{url,hyperref,lineno,microtype,subcaption}
\usepackage{svg}
\usepackage[onehalfspacing]{setspace}
\DeclareUnicodeCharacter{2212}{\ensuremath{-}}

\linenumbers


% Leave a blank line between paragraphs instead of using \\


\def\keyFont{\fontsize{8}{11}\helveticabold }
\def\firstAuthorLast{Weaver {et~al.}} %use et al only if is more than 1 author
\def\Authors{Steven Weaver\,$^{1*}$, Vanessa Davila Conn\,$^{2}$, Hannah Verdonk\,$^{1}$, Joel Wertheim\,$^{3}$, and Sergei L. Kosakovsky Pond\,$^{1}$}
% Affiliations should be keyed to the author's name with superscript numbers and be listed as follows: Laboratory, Institute, Department, Organization, City, State abbreviation (USA, Canada, Australia), and Country (without detailed address information such as city zip codes or street names).
% If one of the authors has a change of address, list the new address below the correspondence details using a superscript symbol and use the same symbol to indicate the author in the author list.
\def\Address{$^{1}$ Center for Viral Evolution, Temple University, Philadelphia, PA, USA \\
$^{2}$ Center for Research in Infectious Diseases, National Institute of Respiratory Diseases, Mexico City, Mexico  \\ 
$^{3}$ Department of Medicine, University of California, San Diego, CA
}
% The Corresponding Author should be marked with an asterisk
% Provide the exact contact address (this time including street name and city zip code) and email of the corresponding author
\def\corrAuthor{Steven Weaver}

\def\corrEmail{sweaver@temple.edu}

\begin{document}
\onecolumn
\firstpage{1}

\title { AUTO-TUNE: FINDING AN OPTIMAL DISTANCE THRESHOLD FOR INFERRING HIV TRANSMISSION CLUSTERS } 

\author[\firstAuthorLast ]{\Authors} %This field will be automatically populated
\address{} %This field will be automatically populated
\correspondance{} %This field will be automatically populated

\extraAuth{}% If there are more than 1 corresponding author, comment this line and uncomment the next one.
%\extraAuth{corresponding Author2 \\ Laboratory X2, Institute X2, Department X2, Organization X2, Street X2, City X2 , State XX2 (only USA, Canada and Australia), Zip Code2, X2 Country X2, email2@uni2.edu}


\maketitle


\begin{abstract}

%%% Leave the Abstract empty if your article does not require one, please see the Summary Table for full details.
\section{}

\noindent Choosing an appropriate distance threshold is an important part of inferring a transmission network to determine the relative growth of clusters within a localized epidemic. This distance threshold determines how close two consensus sequences must be in order for a link to be created between them in the network. Using a distance threshold that is too high can result in a network with many unnecessary links, making it difficult to interpret and analyze. On the other hand, using a distance threshold that is too low can result in a network with too few links, which may not capture key insights into rapidly growing clusters among patients with shared attributes that could benefit from public health intervention measures. 

Here, we present a heuristic scoring approach for tuning a distance threshold by associating each tested threshold against the maximal number of clusters created across all thresholds and the difference between the ratio ($R_{12}$) of the largest cluster in the network to the second largest cluster at each iteration. The number of clusters is normalized between $[0,1]$ then gated via a Gompertz function transform. Meanwhile, the distribution of all $R_{12}$ ratios are converted to $Z$ scores, and normalized relative to the largest positive $Z$ score across all candidate distances. The priority score is the sum of aforementioned two components. 

Published research using the HIV-TRACE software package frequently use the default threshold of $1.5\%$ for HIV pol gene sequences. We apply our scoring heuristic to outbreaks with different characteristics, such as regional or temporal variability, and demonstrate the utility of using the scoring mechanism's suggested distance threshold to identify clusters exhibiting risk factors that would have otherwise been more difficult to identify. For example, while we found that a $1.5\%$ distance threshold is typical for US-like epidemics, recent outbreaks like the CRF07\_BC subtype among men who have sex with men (MSM) in China has been found to have a lower optimal threshold of $0.5\%$ to better capture the transition from injected drug use (IDU) to MSM as the primary risk factor. Alternatively, in communities surrounding Lake Victoria, where there has been sustained transmission for several years, we found that a larger distance threshold is suitable to capture a more risk factor diverse populace with sparse sampling over a longer period of time. Such identification may allow for more informed intervention action by respective public health officials. 

\tiny
  \keyFont{ \section{Keywords:} molecular epidemiology, HIV, network, transmission cluster, surveillance} %All article types: you may provide up to 8 keywords; at least 5 are mandatory.
\end{abstract}

\section{Introduction}

Choosing an appropriate distance threshold is an important part of using a transmission network to track the spread of a contagious disease. This distance threshold determines how close two individuals must be in order for a link to be created between them in the network.

Using a distance threshold that is too small can result in a network with many unnecessary links, making it difficult to interpret and analyze. On the other hand, using a distance threshold that is too large can result in a network with too few links, making it difficult to accurately track the spread of the disease.

To ensure that the transmission network is useful and informative, it is important to carefully consider the appropriate distance threshold. This may vary depending on the specific disease and the context in which it is spreading. For example, a highly contagious respiratory illness may require a smaller distance threshold than a less contagious illness that is primarily spread through direct contact.

In general, the goal is to strike a balance between having enough links to accurately track the spread of the disease, while not having so many links that the network becomes difficult to interpret. This can be achieved through careful analysis and consideration of the specific disease and context.

Overall, choosing an appropriate distance threshold is an important step in using a transmission network to track the spread of a contagious disease. It can help ensure that the network is useful and informative, and can ultimately aid in efforts to control and prevent the spread of the disease.

\section{Methods}

\subsection{Scoring Heuristic Procedure}

Network threshold selection procedure proceeds as follows:

\begin{enumerate}
  \item{For each candidate threshold $d_L$, in increasing order, ranging from the smallest genetic distance in the dataset, up to either the largest distance or a predetermined maximal threshold, we compute two network statistics: $R_{12}$, the ratio of the largest cluster to the second largest cluster, and $C$ – the number of clusters in the network.}

  \item{ A priority score is assigned to each $d_L$. This score measures two properties of the threshold: Does $R_{12}$ jump at $d_L$? How far is the number of clusters $C$ at $d_L$ from the maximal number of clusters over all threshold values? Let there be $N$ overall $d_L$ candidate values, and assume we are examining the ith candidate, $d_L^i$ with $W < i \leq N - W$ ($W$ is a positive integer defined below).

  \begin{enumerate}
    \item{The $R_{12}$ jump is computed by looking at the normalized ratio of the mean $R_{12}$ values computed over the leading window $d_L^{i+1}…d_L^{i+W}$ and the trailing window $d_L^{i-W}… d_L^{i-1}$. The width of the window, $W$, is defined as $(([\frac{N}{100}],3),30)$. The distribution of ratios is converted to $Z$ scores, and normalized relative to the largest positive $Z$ score across all candidate distances, yielding the jump component of the score.}
    \item{The number of clusters, $C_i$ at threshold $d_L^i$ is first normalized to $[0,1]$ through $\frac{{C_{max} - C_i}}{{C_{max} -C_{min}}}$ and next gated via a Gompertz function transform ${1-e}^{-e^{-25x+3}}$ This function provides an ad hoc means for penalizing having too few clusters relative to the maximum over all ranges. For example, a threshold that yields $95\%$ of the maximal number of clusters receives a score of $0.996$, while a threshold that yields $85\%$ - a score of $0.376$.}
    \item{The priority score for $d_L^i$ is the sum of the two components defined in (a) and (b).}
  \end{enumerate}}


  \item{The threshold with the highest priority score will be selected as the suggested automatic distance threshold, if the score is high enough ($1.9$ or more), and either of the two conditions hold.
    \begin{enumerate}
      \item{No other thresholds have priority scores of $1.9$ or higher}
      \item{If other thresholds have priority scores of $1.9$ or higher, then the range of thresholds represented by these options is small (no more than $log(N)$ times the mean step between successive $d_L^i$).}
    \end{enumerate}}

 \item{If no single threshold can be selected in step 3, then the one with the highest priority score is suggested, and an inspection of the plot like the one on the analyze page is recommended to ensure that the threshold is sensible.}
\end{enumerate}

\subsection{Assortativity}

Degree-weighted homophily (DWH) is a measure of similarity between nodes in a network based on their attributes (such as demographic characteristics or behaviors) and their degree (i.e., the number of connections they have to other nodes in the network). It is used to quantify the extent to which nodes with similar attributes tend to be connected to each other more frequently than would be expected by chance.
DWH is calculated as the ratio of the observed number of connections between nodes with similar attributes to the expected number of connections between such nodes, based on their degree.

In mathematical terms, it is defined as:
\begin{equation}
  DWH = \frac{W_M + W_C - 2W_X}{\frac{d_{in}}{nodes_{in}^2} + \frac{d_{out}}{nodes_{out}^2}}
\end{equation}

Where
\begin{itemize}
  \item{$W_M$ : Weight of in-group connections}
  \item{$W_C$ : Weight of out-group connections}
  \item{$W_X$ : Weight of cross-group connections}
  \item{$d_{in}$ : In-group degree}
  \item{$d_{out}$ : Out-group degree}
  \item{$nodes_{in}$ : number of in-group nodes}
  \item{$nodes_{out}$ : number of out-group nodes}
\end{itemize}

DWH ranges from -1 to 1. A DWH value of 0 indicates that there is no more homophily than expected with chance, while a value of 1
indicates that there is perfect homophily (e.g. Birds always link to birds, and only birds). A value of -1 is achieved for perfectly disassortative networks (e.g. Bird never linking with another bird).

DWH is used in social network analysis and in the study of how different
attributes are related to the formation of connections between individuals. It
is used as a way to measure the similarity of attributes between individuals in
a network.

\subsection{Implementation}

The software implementation involves a step-by-step process that utilizes the HIV-TRACE suite of packages. It starts with calculating pairwise distances wth the tn93 tool and a supplied multiple sequence alignment. 
This generated pairwise distances are supplied to the hivnetworkcsv script while providing the -A keyword argument. A brief outline of the software's implementation are as follows

\begin{enumerate}

\item{ Calculate pairwise distances: The user first calculates the pairwise distances using the tn93 fast pairwise distance calculator, providing the necessary threshold value and the input FASTA file. The command for this step is 
\begin{verbatim}
 tn93 -t 0.030 pol.fasta > pairwise_distances.15.tn93.csv
\end{verbatim}

Please note that the threshold should include the maximal range one is intending to test.
}

\item {Compute distance threshold scores: The hivnetworkcsv script is then executed with the required input file, format, and autotune option to generate a tab-separated output file, as shown below
\begin{verbatim}
 hivnetworkcsv -i pairwise_distances.15.tn93.csv -f plain -A 0 > autotune_report.tsv
\end{verbatim}
}

\item {Visualize the report: Users can upload the generated autotune\_report.tsv file to http://autotune.datamonkey.org/analyze for visualization and further analysis of the data. This web-based platform provides an interactive environment to explore scores and other metrics across the range of tested outputs. }

\item {Run HIV-TRACE: Once AUTO-TUNEd threshold(s) are settled upon after review, the user runs the HIV-TRACE command with the appropriate input FASTA file, distance threshold, and other required arguments. The output is saved as a JSON file. An example command is
	\begin{verbatim}
 		hivtrace -i ./INPUT.FASTA -a resolve -r HXB2_prrt -t < autotune_threshold > -m 500 -g .05 > hivtrace.results.json
	\end{verbatim}
}

\subsubsection{Optional : Compute Assortatviity Metrics}

\item{ Annotate results: The hivnetworkannotate script is used to annotate the results obtained from the HIV-TRACE step with attributes. The script takes the JSON results file, node attributes file, schema file, and a resolve flag as input.
	\begin{verbatim}
 		hivnetworkannotate -n hivtrace.results.json -a node_attributes.json -g schema.json -r
	\end{verbatim}
 For more information, users can refer to the hivnetworkannotate documentation.
}

\item{ Analyze the results with DWH: After the results file has been annotated, the user can proceed to the assortativity page, http://autotune.datamonkey.org/assortativity, for further analysis of the output. }

AUTO-TUNE is readily accessible on GitHub as part of the hivclustering repository (https://github.com/veg/hivclustering). It is integrated into the command-line interface of the software as the -A or --auto-profile argument. hivclustering is a key component of the HIV-TRACE suite of tools, a resource for the inference, analysis and visualization of HIV transmission networks. 

The Degree Weighted Homophily (DWH) calculation tool, an integral component of the assortativity step, is developed using TypeScript, a statically typed superset of JavaScript that ensures robustness and scalability. In an effort to promote accessibility and ease of integration, the DWH tool is packaged and distributed through the Node Package Manager (NPM), enabling researchers and developers to conveniently incorporate this advanced analytical tool into their own projects and workflows. DWH can be used in-browser or as a command line tool, allowing researchers and developers to employ the tool in an interactive command-line interface or integrate it into larger software applications, thus catering to a diverse array of technical needs and preferences. Instructions for usage and installation is found on Github (https://github.com/veg/dwh).

\end{enumerate}

The described workflow offers a systematic approach to analyze potential distance thresholds for one's data with AUTO-TUNE, from calculating pairwise distances to visualizing and annotating results.

\subsection{Visualization}

Visualizations of AUTO-TUNE results are accessible at http://autotune.datamonkey.org/analyze. It is a dynamic and interactive web-based platform that offers visualization and analysis of results generated by AUTO-TUNE. 
The website provides a comprehensive view of the data by generating various plots across candidate distance thresholds. These include a score plot, allowing users to identify trends and anomalies across the full range of thresholds. Additionally, it generates a graph showing the number of clusters across candidate thresholds, one of the components that contribute to the score. The site also includes an R1/R2 plot that displays the ratio of the largest cluster to the second largest cluster across candidate thresholds, which is the other metric that contributes to the scoring heuristic. 

An assortativity tool is available at http://autotune.datamonkey.org/assortativity, and is an advanced analytical tool engineered to facilitate the calculation of Degree Weighted Homophily (DWH) values. It utilizes the DWH NPM package to generate a tabular representation of DWH values corresponding to each value for a selected attribute annotation, providing an exhaustive examination of the interrelationships for the field.
A notable feature is the computation of the panmictic range, which involves a label permutation test to generate the null distribution of DWH values. This feature establishes a comparative baseline that aids in determining the significance of homophily versus what would be expected by chance.
Lastly, the site also provides a plot of the fraction of pairwise connections, normalized by degree, for each value pertinent to the selected field. This visual depiction facilitates an intuitive comprehension of the distribution and interconnections within the dataset.

The site aims to offer a user-friendly interface for data visualization, playing an important role in interpreting and understanding AUTO-TUNE's output data. The visualization code is available on Github (https://github.com/stevenweaver/autotune-app/).

\subsection{Data curation and analysis}

\section{Results}

\subsection{Case Study 1: Middle Tennessee, Seattle, and Alberta}

\subsection{Case Study 2: Comparison with Prior Publications}

\subsection{Case Study 3: Comparison with Prior Publications}


\section{Discussion}

\section*{Conflict of Interest Statement}
%All financial, commercial or other relationships that might be perceived by the academic community as representing a potential conflict of interest must be disclosed. If no such relationship exists, authors will be asked to confirm the following statement: 

The authors declare that the research was conducted in the absence of any commercial or financial relationships that could be construed as a potential conflict of interest.

\section*{Author Contributions}

The Author Contributions section is mandatory for all articles, including articles by sole authors. If an appropriate statement is not provided on submission, a standard one will be inserted during the production process. The Author Contributions statement must describe the contributions of individual authors referred to by their initials and, in doing so, all authors agree to be accountable for the content of the work. Please see  \href{https://www.frontiersin.org/guidelines/policies-and-publication-ethics#authorship-and-author-responsibilities}{here} for full authorship criteria.

\section*{Funding}
Details of all funding sources should be provided, including grant numbers if applicable. Please ensure to add all necessary funding information, as after publication this is no longer possible.

\section*{Acknowledgments}
This is a short text to acknowledge the contributions of specific colleagues, institutions, or agencies that aided the efforts of the authors.

\section*{Supplemental Data}
 \href{https://www.frontiersin.org/guidelines/author-guidelines#supplementary-material}{Supplementary Material} should be uploaded separately on submission, if there are Supplementary Figures, please include the caption in the same file as the figure. LaTeX Supplementary Material templates can be found in the Frontiers LaTeX folder.

\section*{Data Availability Statement}
The datasets [GENERATED/ANALYZED] for this study can be found in the [NAME OF REPOSITORY] [LINK].
% Please see the availability of data guidelines for more information, at https://www.frontiersin.org/guidelines/policies-and-publication-ethics#materials-and-data-policies

\bibliographystyle{Frontiers-Harvard} %  Many Frontiers journals use the Harvard referencing system (Author-date), to find the style and resources for the journal you are submitting to: https://zendesk.frontiersin.org/hc/en-us/articles/360017860337-Frontiers-Reference-Styles-by-Journal. For Humanities and Social Sciences articles please include page numbers in the in-text citations 
%\bibliographystyle{Frontiers-Vancouver} % Many Frontiers journals use the numbered referencing system, to find the style and resources for the journal you are submitting to: https://zendesk.frontiersin.org/hc/en-us/articles/360017860337-Frontiers-Reference-Styles-by-Journal
\bibliography{test}
\nocite{*}

%%% Make sure to upload the bib file along with the tex file and PDF
%%% Please see the test.bib file for some examples of references


\begin{table}[ht]
\caption{Homophily Comparison}
\centering
\label{tab:homophily}
\begin{tabular}{|c|c|c|}
\hline
Dataset & Art's Threshold Avg. Homophily & AUTO-TUNE Avg. Homophily \\
\hline
TN & 0.0079 & 0.0147 \\
Seattle & 0.0259 & 0.0348 \\
Canada & -0.0536 & -0.0448 \\
\hline
\end{tabular}
\end{table}

\begin{table}[ht]
\caption{Paper Comparison}
\centering
\label{tab:paperComparison}
\begin{tabular}{|c|c|c|c|c|}
\hline
PMID & Country & Collection Date & Threshold Used & AUTO-TUNE \\
\hline
29975689 & South Africa & 2011-2015 & 2.5\% & 2.584\% \\
30574123 & USA & 1997-2008 & 2\% & 1.848\% \\
32500089 & China & 2008-2015 & 0.5\% & 0.675\% \\
32693608 & Uganda & 2009-2016 & 1.5\% & 1.707\% \\
33281803 & China & 2000-2016 & 0.5\%/0.7\% & 0.676\% \\
33901684 & China & 2008-2012 & 1.5\% & 1.215\% \\
34452506 & Canada & 1996-2017 & 1.5\%/2.5\% & 0.547\% \\
31041344 & USA & 1997-2017 & 1.5\% & 0.927\% \\
\hline
\end{tabular}
\end{table}



\end{document}
